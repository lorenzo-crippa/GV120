\documentclass[xcolor=table]{beamer}

\usepackage{graphicx}
\usepackage{wrapfig}
\usepackage{url}

\title{GV120 - Politics and Economics Policies}
\subtitle{University of Essex - Department of Government}
\date{Week 4 -- 25 October, 2019}				% or you can specify a date, just write it down instead of "\today"
\author{Lorenzo Crippa} 

\usetheme[progressbar=frametitle]{metropolis}
\usecolortheme{seahorse}						% try others: wolverine; crane...

\begin{document}
\frame{
\titlepage
}

\frame{
\frametitle{Change office hour}
\begin{center}
Office 5B.153 \\
Wednesday 14:00 to 16:00 \\
l.crippa@essex.ac.uk
\end{center}
}

\frame{
\frametitle{Seminar timetable}

\begin{table}[]
\centering
\resizebox{8.5cm}{!}{%
\begin{tabular}{|c|l|}
\hline
{\color[HTML]{333333} \textbf{Week number}} & \multicolumn{1}{c|}{{\color[HTML]{333333} \textbf{Groups and activities}}}                                        \\ \hline
{\color[HTML]{333333} 4}                    & {\color[HTML]{333333} \begin{tabular}[c]{@{}l@{}}Ester Cavaioni and Blake Mallon\end{tabular}}                                                                           \\ \hline
{\color[HTML]{333333} 5}                    & {\color[HTML]{333333} \begin{tabular}[c]{@{}l@{}}Emma Paulinyova and Emma Sornlund\\ Sònia Villalba and Sally Touray\end{tabular}}                                                                           \\ \hline
{\color[HTML]{333333} 6}                    & {\color[HTML]{333333} \begin{tabular}[c]{@{}l@{}}Discuss take-home assignment\\Samuel Leonard and Joshua Kelly\end{tabular}} \\ \hline
{\color[HTML]{333333} 7}                    & {\color[HTML]{333333} \begin{tabular}[c]{@{}l@{}}Discuss in-lecture test\\Halide Asafogullari and Azhan Airwan (week 6)\end{tabular}}      \\ \hline
{\color[HTML]{333333} 8}                    & {\color[HTML]{333333} \begin{tabular}[c]{@{}l@{}}Henry Adebiyi and Nyima Jobe (week 6)\\Kwamina Keelson and Shiv Bhatt (week 7)\end{tabular}}                                                      \\ \hline
{\color[HTML]{333333} 9}                    & {\color[HTML]{333333} \begin{tabular}[c]{@{}l@{}}Ethan Liddel\\\end{tabular}}                                                                           \\ \hline
{\color[HTML]{333333} 10}                   & {\color[HTML]{333333}  \begin{tabular}[c]{@{}l@{}}Dorsa Heidari and Aleksandra Waszescik\\Domantas Seveliovas and Abigail Kiely\end{tabular}}                                                                           \\ \hline
{\color[HTML]{333333} 11}                   & {\color[HTML]{333333} Amine Yemmou (week 10)}                                                      \\ \hline
\end{tabular}%
}
\end{table}
}

\frame{
\frametitle{The Tragedy of the Commons}

Hardin, Garrett (1968). The Tragedy of the Commons. \emph{Science 162} (3859), 1243-1248. 


Main argument: The population problem has no technical solution. \\
A ``technical solution'' does not involve moral choices of sort.

}

\frame{
\frametitle{The population problem}
\begin{itemize}
\item What is the population problem?
	\begin{itemize}
	\item[--] Population grows exponentially. Resources are finite
	\item[--] Malthus' \emph{Principles of Political Economy} (1820)
	\end{itemize}
\item Why can't it have a technical solution?
	\begin{itemize}
	\item[--] Because of the Tragedy of the Commons
	\item[--] A technical solution (economic rationality) leads to tragedy
	\end{itemize}
\end{itemize}
}

\frame{
\frametitle{The tragedy of the commons}
\begin{itemize}
\item Quintessential examples (common field, oceans, fishing and polluting)
\item Commons are \emph{not} pure public goods
	\begin{itemize}
	\item[--] They are non-excludable
	\item[--] They are a finite resource, thus they are (partly) rival
	\item[--] Public costs and private benefits
	\end{itemize}
\item A strong argument against A. Smith's reliance on private interest (commons are no market!)
\end{itemize}
}

\frame{
\frametitle{From ToC to the population problem}
\begin{itemize}
\item Freedom to overbreed (population grows exponentially)
\item Given limited resources, a tragedy is inevitable
\item ``Freedom to breed is intolerable'' (Hardin 1968, 1246)
\end{itemize}
}

\frame{
\frametitle{What solutions to the ToC?}
\begin{enumerate}
\item Privatise them: Back to A. Smith
\item Introduce mutually agreed-upon coercion (Hardin)
\item Rely on self-organisation 
	\begin{itemize}
	\item[--] ``Conscience is Self-Eliminating'' (Hardin 1968, 1246) in the long run and produces anxiety in the short run
	\item[--] Self-organisation can emerge naturally at the level of small communities (Ostrom 2003)
	\end{itemize}
\end{enumerate}
}

\frame{
\frametitle{Conclusion: Two questions to discuss}
\begin{itemize}
\item What solution is just?
\item What solution is practical?  
\end{itemize}
}

\frame{
\frametitle{Further suggested readings}
On the differences between Public Good Games and Prisoners' Dilemma Games: \\
\begin{enumerate}
\item Conybeare, John A. C. (1984). Public Goods, Prisoners' Dilemmas and the International Political Economy. \emph{International Studies Quarterly 28} (1), 5-22.
\end{enumerate}

On the introduction of relational payoffs for agents: \\
\begin{enumerate}
\item[2.] Grieco, Joseph M. (1988). Realist Theory and the Problem of International Cooperation: Analysis with an Amended Prisoner's Dilemma Model. \emph{The Journal of Politics 50} (3), 600-624.
\end{enumerate}

On the benefits of sustained interactions over time: \\
\begin{enumerate}
\item[3.] Axelrod, Robert (1984). \emph{The Evolution of Cooperation}. Basic Books, New York.
\end{enumerate}
}

\frame{
\frametitle{References}

Dietz, Thomas, Elinor Ostrom, and Paul C. Stern (2003). The struggle to govern the commons. \emph{Science 302} (5652), 1907-1912.

Hardin, Garrett (1968). The Tragedy of the Commons. \emph{Science 162} (3859), 1243-1248.

Malthus, Thomas (1989) [1820]. \emph{Principles of Political Economy}. Edited by: John Pullen. Cambridge University Press.
}



\end{document}
