\documentclass[xcolor=table,dvipsnames]{beamer}

\usepackage{graphicx}
\usepackage{wrapfig}
\usepackage{url}

\title{GV120 - Politics and Economics Policies5}
\subtitle{University of Essex - Department of Government}
\date{Week 7 -- 15 November, 2019}				
\author{Lorenzo Crippa} 

\usetheme[progressbar=frametitle]{metropolis}
\usecolortheme{seahorse}						% try others: wolverine; crane...

\begin{document}
\frame{
\titlepage
}

\frame{
\frametitle{Communication}
\begin{center}
Week 8: Office hour on Thursday, from 11 to 13 \\
({\bf NOT} Wednesday 14 to 16) \\
Office 5B.153 \\
l.crippa@essex.ac.uk
\end{center}
}


\frame{
\frametitle{Seminar timetable}

\begin{table}[]
\centering
\resizebox{8.5cm}{!}{%
\begin{tabular}{|c|l|}
\hline
{\color[HTML]{333333} \textbf{Week number}} & \multicolumn{1}{c|}{{\color[HTML]{333333} \textbf{Groups and activities}}}                                        \\ \hline
{\color[HTML]{333333} 7}                    & {\color[HTML]{333333} \begin{tabular}[c]{@{}l@{}}Discuss in-lecture test\\Azhan Airwan (week 6)\end{tabular}}      \\ \hline
{\color[HTML]{333333} 8}                    & {\color[HTML]{333333} \begin{tabular}[c]{@{}l@{}}Henry Adebiyi and Nyima Jobe (week 6)\\Kwamina Keelson and Shiv Bhatt (week 7)\end{tabular}}                                                      \\ \hline
{\color[HTML]{333333} 9}                    & {\color[HTML]{333333} \begin{tabular}[c]{@{}l@{}}Ethan Liddel and Patrick Turuthi \\ Halide Asafogullari (week 6)\end{tabular}}                                                                           \\ \hline
{\color[HTML]{333333} 10}                   & {\color[HTML]{333333}  \begin{tabular}[c]{@{}l@{}}Dorsa Heidari and Aleksandra Waszescik\\Domantas Seveliovas and Abigail Kiely\end{tabular}}                                                                           \\ \hline
{\color[HTML]{333333} 11}                   & {\color[HTML]{333333} \begin{tabular} [c]{@{}l@{}}Amine Yemmou and Khashayar \\ Namdari-Gharaghani (week 10)\end{tabular}}                                                      \\ \hline
\end{tabular}%
}
\end{table}
}


\frame{
\frametitle{In-lecture test indications}
\begin{center}
A selection of the indications from previous year, by Dietrich.
\textbf{Read} Dietrich's slides on Moodle, in any case
\end{center}
}

\frame{
\frametitle{General advices for your in-lecture test}
Read the entire question

\begin{itemize}
\item Read a question in its entirety before glancing over the answer options
\item You might be required to select the TRUE answer or the FALSE one
\item do not jump straight to what you think is the most logical answer
\item Read each option thoroughly before choosing one
\end{itemize}
}

\frame{
\frametitle{If you know the correct answer}
If you are sure about the correct answer, answer the question in your mind first, before reading the options, then:
\begin{enumerate}
\item Always make sure that you read all other answers anyway
\item Remember to choose \emph{the} best answer
\item Do not \emph{assume} you know the right answer
\end{enumerate}
}

\frame{
\frametitle{If you do not know the correct answer}
If you do not know the correct answer, or you are unsure about it, you can:
\begin{enumerate}
\item Eliminate those that you know are wrong
\item Focus on the remaining ones
\item When in absolute doubt, make an educated guess (if you are not subtracted points for mistakes)
\end{enumerate}
}

\frame{
\frametitle{Focus on THE best answer}
\begin{center}
Remember: if you are asked for \emph{the} best answer you do not only need an answer to be correct, but also to be the best one. \\ Often many answers will seem correct, but there is \emph{one} best answer to the question
\end{center}
}

\frame{
\frametitle{Time management}
As general advices on time management:
\begin{itemize}
\item Answer the questions you know first
\item Do not get stuck in questions you are not sure of
\item After having completed all you are sure of, go back to what you left behind
\end{itemize}
}

\frame{
\frametitle{Watch out}
Pay attention to the following expressions:
\begin{itemize}
\item Adverbs like sometimes, always and never. Always and never imply no counterfactual
\item ``All of the above'' and ``None of the above'' should not be picked, respectively, if you are sure that at least one of the answers provided is incorrect and if at least one of the answers is correct. 
\item When two answers are correct in a multiple choice question with ``All of the above'', then that is probably the correct choice
\end{itemize}
}

\frame{
\frametitle{Mock test with examples from your presentations}
\begin{enumerate}
\item A group of north-eastern states in the U.S. caps carbon emissions from their power plants. Choose the BEST answer in the following list
	\begin{itemize}
	\item[a.] The good provided is a club good
	\item[b.] The good provided is partly rival but non-excludable
	\item[c.] The good provided is a public good, because it is non rival and non excludable
	\item[d.] The good provided is non excludable
	\end{itemize}
\end{enumerate}
}

\frame{
\frametitle{Mock test with examples from your presentations}
\begin{enumerate}
\item A group of north-eastern states in the U.S. caps carbon emissions from their power plants. Choose the BEST answer in the following list
	\begin{itemize}
	\item[a.] The good provided is a club good
	\item[b.] The good provided is partly rival but non-excludable
	\item[c.] \textcolor{ForestGreen}{The good provided is a public good, because it is non rival and non excludable}
	\item[d.] The good provided is non excludable
	\end{itemize}
\end{enumerate}
}

\frame{
\frametitle{Mock test with examples from your presentations}
\begin{enumerate}
\item[2.] Examples of commons include a public field for sheep-breeding, the freedom of humans to breed, Passenger Pigeons (before their extinctions), Gulf of Mexico dead zone. Which of the following is INCORRECT?
	\begin{itemize}
	\item[a.] Commons are partly rival because they are finite goods
	\item[b.] Commons are rival because they are enclosed
	\item[c.] Commons are non-excludable because it is not possible to impede anyone from consuming them
	\item[d.] One solution to the tragedy of the Commons is the exercise of coercion
	\end{itemize}
\end{enumerate}
}

\frame{
\frametitle{Mock test with examples from your presentations}
\begin{enumerate}
\item[2.] Examples of commons include a public field for sheep-breeding, the freedom of humans to breed, Passenger Pigeons (before their extinctions), Gulf of Mexico dead zone. Which of the following is INCORRECT?
	\begin{itemize}
	\item[a.] Commons are partly rival because they are finite goods
	\item[b.] \textcolor{ForestGreen}{Commons are rival because they are enclosed}
	\item[c.] Commons are non-excludable because it is not possible to impede anyone from consuming them
	\item[d.] One solution to the tragedy of the Commons is the exercise of coercion
	\end{itemize}
\end{enumerate}
}


\frame{
\frametitle{Mock test with examples from your presentations}
\begin{enumerate}
\item[3.] Consider the Global Polio Eradication Initiative (1988). Choose the TRUE answer among the following ones:
	\begin{itemize}
	\item[a.] The best aggregation technology to provide the good is summation
	\item[b.] The best aggregation technology to provide the good is best shot
	\item[c.] The best aggregation technology to provide the good is weakest link
	\item[d.] No technology can aggregate this good efficiently
	\end{itemize}
\end{enumerate}
}

\frame{
\frametitle{Mock test with examples from your presentations}
\begin{enumerate}
\item[3.] Consider the Global Polio Eradication Initiative (1988). Choose the TRUE answer among the following ones:
	\begin{itemize}
	\item[a.] The best aggregation technology to provide the good is summation
	\item[b.] The best aggregation technology to provide the good is best shot
	\item[c.] \textcolor{ForestGreen}{The best aggregation technology to provide the good is weakest link}
	\item[d.] No technology can aggregate this good efficiently
	\end{itemize}
\end{enumerate}
}

\frame{
\frametitle{Mock test from Olson's principles}
\begin{enumerate}
\item[4.] Which of the following is NOT one of Olson's 5 principles of collective action favouring public good provisions?
	\begin{itemize}
	\item[a.] Groups with similar preferences are more likely to form.
	\item[b.] Small groups of states lead to better provision.
	\item[c.] If stakes are high, states are compelled to contribute.
	\item[d.] States that share borders are more likely to contribute.
	\end{itemize}
\end{enumerate}
}

\frame{
\frametitle{Mock test from Olson's principles}
\begin{enumerate}
\item[4.] Which of the following is NOT one of Olson's 5 principles of collective action favouring public good provisions?
	\begin{itemize}
	\item[a.] Groups with similar preferences are more likely to form.
	\item[b.] Small groups of states lead to better provision.
	\item[c.] If stakes are high, states are compelled to contribute.
	\item[d.] \textcolor{ForestGreen}{States that share borders are more likely to contribute.}
	\end{itemize}
\end{enumerate}
}

\frame{
\frametitle{Conclusion}
\begin{center}
Questions? Doubts? 

Good luck and see you next week!
\end{center}
}
\end{document}