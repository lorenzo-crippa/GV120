\documentclass[xcolor=table]{beamer}

\usepackage{graphicx}
\usepackage{wrapfig}
\usepackage{url}

\title{GV120 - Politics and Economics Policies}
\subtitle{University of Essex - Department of Government}
\date{Week 5 -- 1 November, 2019}				% or you can specify a date, just write it down instead of "\today"			
\author{Lorenzo Crippa} 

\usetheme[progressbar=frametitle]{metropolis}
\usecolortheme{seahorse}						% try others: wolverine; crane...

\begin{document}
\frame{
\titlepage
}


\frame{
\frametitle{Seminar timetable}

\begin{table}[]
\centering
\resizebox{8.5cm}{!}{%
\begin{tabular}{|c|l|}
\hline
{\color[HTML]{333333} \textbf{Week number}} & \multicolumn{1}{c|}{{\color[HTML]{333333} \textbf{Groups and activities}}}                                        \\ \hline
{\color[HTML]{333333} 5}                    & {\color[HTML]{333333} \begin{tabular}[c]{@{}l@{}}Sònia Villalba and Sally Touray \\ Emma Paulinyova and Emma Sornlund \end{tabular}}                                                                           \\ \hline
{\color[HTML]{333333} 6}                    & {\color[HTML]{333333} \begin{tabular}[c]{@{}l@{}}Discuss take-home assignment\\Samuel Leonard and Joshua Kelly\end{tabular}} \\ \hline
{\color[HTML]{333333} 7}                    & {\color[HTML]{333333} \begin{tabular}[c]{@{}l@{}}Discuss in-lecture test\\Halide Asafogullari and Azhan Airwan (week 6)\end{tabular}}      \\ \hline
{\color[HTML]{333333} 8}                    & {\color[HTML]{333333} \begin{tabular}[c]{@{}l@{}}Henry Adebiyi and Nyima Jobe (week 6)\\Kwamina Keelson and Shiv Bhatt (week 7)\end{tabular}}                                                      \\ \hline
{\color[HTML]{333333} 9}                    & {\color[HTML]{333333} \begin{tabular}[c]{@{}l@{}}Ethan Liddel and Patrick Turuthi \\\end{tabular}}                                                                           \\ \hline
{\color[HTML]{333333} 10}                   & {\color[HTML]{333333}  \begin{tabular}[c]{@{}l@{}}Dorsa Heidari and Aleksandra Waszescik\\Domantas Seveliovas and Abigail Kiely\end{tabular}}                                                                           \\ \hline
{\color[HTML]{333333} 11}                   & {\color[HTML]{333333} \begin{tabular} [c]{@{}l@{}}Amine Yemmou and Khashayar \\ Namdari-Gharaghani (week 10)\end{tabular}}                                                      \\ \hline
\end{tabular}%
}
\end{table}
}

\frame{
\frametitle{Today's presentations}
\begin{center}
Time for Sònia Villalba and Sally Touray and then for \\ Emma Paulinyova and Emma Sornlund to present
\end{center}
}

\frame{
\frametitle{Recap of concepts learnt in class}
Remember:
\begin{itemize}
\item[a.] Definition of pure public good
	\begin{itemize}
	\item[--] 
	\item[--] 
	\end{itemize}
\item[b.] Four aggregation technologies
	\begin{itemize}
	\item[--] 
	\item[--] 
	\item[--] 
	\item[--] 
	\end{itemize}
\item[c.] Five Olson's principles of collective action (Olson 1965)
	\begin{itemize}
	\item[--] 
	\item[--] 
	\item[--] 
	\item[--] 
	\item[--] 
	\end{itemize}
\end{itemize}
}

\frame{
\frametitle{Recap of concepts learnt in class}
Remember:
\begin{itemize}
\item[a.] Definition of pure public good
	\begin{itemize}
	\item[--] Non-rivalry
	\item[--] Non-excludability
	\end{itemize}
\item[b.] Four aggregation technologies
	\begin{itemize}
	\item[--] Summation
	\item[--] Weakest link
	\item[--] Threshold
	\item[--] Best shot
	\end{itemize}
\item[c.] Five Olson's principles of collective action (Olson 1965)
	\begin{itemize}
	\item[--] Similar preferences
	\item[--] Differences in size
	\item[--] Small groups
	\item[--] High stakes
	\item[--] Private benefits
	\end{itemize}
\end{itemize}
}

\frame{
\frametitle{Apply concepts learnt in class to your examples}
\begin{enumerate}
\item A group of north-eastern states in the U.S. capping carbon emissions from their power plants
	\begin{itemize}
	\item[a.] Is that a pure public good?
	\item[b.] What aggregation technology is in place?
	\item[c.] Can Olson's 5 principles help solve the problem of collective action?
	\end{itemize}
\end{enumerate}
}

\frame{
\frametitle{Apply concepts learnt in class to your examples}
\begin{enumerate}
\item[2.] Commons (\emph{e.g.}: freedom to breed, extinction of Passenger Pigeons, Gulf of Mexico dead zone)
	\begin{itemize}
	\item[a.] Are commons a pure public good?
	\item[b.] What aggregation technology is in place to preserve them?
	\item[c.] Can Olson's 5 principles help solve the problem of collective action?
	\end{itemize}
\end{enumerate}
}

\frame{
\frametitle{Apply concepts learnt in class to your examples}
\begin{enumerate}
\item[3.] Global Polio Eradication Initiative (1988)
	\begin{itemize}
	\item[a.] Is that a pure public good?
	\item[b.] What aggregation technology is in place?
	\item[c.] Can Olson's 5 principles help solve the problem of collective action?
	\end{itemize}
\end{enumerate}
}

\frame{
\frametitle{Apply concepts learnt in class to your examples}
\begin{enumerate}
\item[4.] Global warming protection
	\begin{itemize}
	\item[a.] Is that a pure public good?
	\item[b.] What aggregation technology is in place?
	\item[c.] Can Olson's 5 principles help solve the problem of collective action?
	\end{itemize}
\end{enumerate}
}

\frame{
\frametitle{References}
Barett, Scott (2007). ``Financing and burden sharing: Paying for global public goods''. In: Barrett, Scott (Ed.). \emph{Why Cooperate?: The incentive to Supply Global Public Goods}. Oxford University Press.

Kaul, Inge and Katel Le Goulven (2003). ``Institutional Options for Producing Global Public Goods''. In: Kaul, Inge \emph{et al.} (Eds.). \emph{Providing Global Public Goods: Managing Globalization}. Oxford University Press.

Olson, Mancur (1965). \emph{The Logic of Collective Action}. Cambridge, MA: Harvard University Press}
\end{document}
