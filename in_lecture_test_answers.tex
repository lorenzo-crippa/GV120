\documentclass[xcolor=table,dvipsnames]{beamer}

\usepackage{graphicx}
\usepackage{wrapfig}
\usepackage{url}

\title{GV120 - Politics and Economics Policies}
\subtitle{University of Essex - Department of Government}
\date{Week 8 -- 22 November, 2019}				
\author{Lorenzo Crippa} 

\usetheme[progressbar=frametitle]{metropolis}
\usecolortheme{seahorse}						% try others: wolverine; crane...

\begin{document}
\frame{
\titlepage
}

\frame{
\frametitle{In-lecture test}
\begin{center}
Correction of the in-lecture test
\end{center}
}

\frame{
\frametitle{Question 1}
What is Pareto efficiency (also known as Pareto optimality)? [Please check the one correct answer.]
\begin{enumerate}
\item An allocation of resources where no one can be made better off without someone being made worse off.
\item An allocation of resources where we can make someone better off without making someone else worse off.
\item An allocation of resources that reduce inequality.
\item An allocation of scarce resources.
\end{enumerate} 
\pause
Correct answer: 1
}

\frame{
\frametitle{Question 2}
Which of the following best describes a negative externality? [Please check the one correct answer.]
\begin{enumerate}
\item A transnational issue with negative consequences.
\item When actions by one individual impose a cost on others but the individual does not compensate them.
\item When actions by one individual impose a cost on others and the individual compensates them.
\item When one individual confers benefits upon others but does not reap a reward.	
\end{enumerate} 
\pause
Correct answer: 2
}


\frame{
\frametitle{Question 3}
What is adverse selection (also known as `hidden type')? [Please check the one correct answer.]
\begin{enumerate}
\item When a market participant does something the other might not observe.
\item When there is a failure of competition.
\item When a market participant knows something the other might not know.
\item When a market participant acts in a risky manner because they do not expect to pay the consequences.
\end{enumerate} 
\pause
Correct answer: 3
}

\frame{
\frametitle{Question 4}
What is a pure public good? [Please check the one correct answer.]
\begin{enumerate}
\item A good provided by the government.
\item A good that is organic.
\item A good that is very difficult to exclude people from using and, when used, does not reduce availability to others.
\item A good that confers private benefits to its users and can be consumed simultaneously by more than one person.
\end{enumerate} 
\pause
Correct answer: 3
}

\frame{
\frametitle{Question 5}
What is the main problem with non-excludable public goods? [Please check the one correct answer.]
\begin{enumerate}
\item Consumers will free-ride on the provision of others.
\item Individuals are irrational.
\item You cannot make anyone better off without someone being made worse off.
\item It costs nothing for an additional individual to enjoy benefits.\end{enumerate} 
\pause
Correct answer: 1
}

\frame{
\frametitle{Question 6}
What does it mean when we say a public good is non-rival? [Please check the one correct answer.]
\begin{enumerate}
\item Consumers will free-ride on the provision of others.
\item If consumed, no other consumer can benefit from the public good.
\item Consumption by one individual will not reduce the good's availability to others.
\item The higher the number of consumers of the good, the more everyone else consumes.
\item Providers will oversupply the good.
\end{enumerate} 
\pause
Correct answer: 3
}

\frame{
\frametitle{Question 7}
What are the two classic characteristics of a ``Club''- type public good? [Please check two answers.]
\begin{enumerate}
\item Excludable.
\item Non-rival.
\item Fully or partially rival.
\item Weakest link.
\item Non-excludable.
\end{enumerate} 
\pause
Correct answers: 1 - 3
}

\frame{
\frametitle{Question 8}
What is the general definition of a ``joint product'' public good? [Please check the one correct answer.]
\begin{enumerate}
\item A club good and a pure public good joined together.
\item When multiple types of public goods are bundled together.
\item A public good with weakest-link dynamics.
\item Two products sold together.
\end{enumerate} 
\pause
Correct answer: 2
}

\frame{
\frametitle{Question 9}
What is the difference between a pure public good and a commons? [Please check all applicable answers.]
\begin{enumerate}
\item A pure public good is provided by the government and the commons exists in nature.
\item The benefits from using the commons are private, whereas the benefits from consuming a pure public good are shared by the public.
\item The costs of using the commons are shared, whereas the costs from consuming a pure public good are private.
\item There is no difference between them.
\end{enumerate} 
\pause
Correct answers: 2 - 3
}

\frame{
\frametitle{Question 10}
What is the most likely outcome of the classic game ``Prisoner's Dilemma'' game? [Please check the one correct answer.]
\begin{enumerate}
\item Both players choose to cooperate.
\item One player chooses to cooperate, the other chooses not to cooperate.
\item Both players choose not to cooperate with each other.
\item One player does not make a choice.
\item Neither player makes a choice.
\item Both players consciously choose to make the worst possible decision.
\end{enumerate} 
\pause
Correct answer: 3
}

\frame{
\frametitle{Question 11}
What are Garrett Hardin's solutions to the Tragedy of the Commons? [Please check all correct answers.]
\begin{enumerate}
\item People will eventually develop rules on how to manage common resources on their own.
\item There are no solutions because people are trapped in a commons dilemma.
\item Private property.
\item Government administration of commons usage.
\end{enumerate} 
\pause
Correct answers: 3 - 4
}

\frame{
\frametitle{Question 12}
What best describes the underlying core problem of collective action failures? [Please check the one correct answer.]
\begin{enumerate}
\item The group interest is in conflict with what is good for society.
\item The individual has incomplete knowledge.
\item Public space can be congested.
\item Rational decisions of individuals lead to suboptimal collective outcomes.
\item The group interest is aligned with the individual interest.
\end{enumerate} 
\pause
Correct answer: 4
}

\frame{
\frametitle{Question 13}
Which of the following defines the exploitation hypothesis? [Please check the one correct answer.]
\begin{enumerate}
\item Larger (rich) members carry a burden for the smaller (poor) players.
Small players will free-ride.
\item Smaller (poor) members carry a burden for the larger (rich) players. Large
players will free-ride.
\end{enumerate} 
\pause
Correct answer: 1
}

\frame{
\frametitle{Question 14}
Which of the following is NOT one of Mancur Olson's principles of collective
action? [Please check the one correct answer.]
\begin{enumerate}
\item If stakes are high members will be compelled to contribute.
\item Groups with `large' and `small' members lead to suboptimal provision.
\item Groups with similar preferences more likely to form.
\item Small groups lead to better provision.
\end{enumerate} 
\pause
Correct answer: 2
}

\frame{
\frametitle{Question 15}
What type of public good is the Panama Canal? [Please check the one correct answer.]
\begin{enumerate}
\item Impurely public good: some rivalry but no exclusion.
\item Club good.
\item Joint product.
\item Open access commons. 
\end{enumerate}
\pause
Correct answer: 2
}

\frame{
\frametitle{Question 16}
What is the best definition of the aggregation technology of a public good? [Please check the one correct answer.]
\begin{enumerate}
\item The way some public goods are easier supplied than others.
\item The way weakest-link public goods lead to the a low level of provision.
\item The way threshold public goods lead to the required quantity of a public good.
\item The way best-shot public goods lead to oversupply of a public good.
\item The way individual contributions lead to the overall provided level of
a public good.
\end{enumerate} 
\pause
Correct answer: 5
}

\frame{
\frametitle{Question 17}
What aggregation technology is associated with security across a computer
network? [Please check the one correct answer.]
\begin{enumerate}
\item Best-shot.
\item Summation.
\item Rivalry.
\item Threshold.
\item Weakest-link.
\end{enumerate} 
\pause
Correct answer: 5
}

\frame{
\frametitle{Question 18}
Which of the following goods is rival but non-excludable? [Please check the one correct answer.]
\begin{enumerate}
\item Commons.
\item Pure public good.
\item Club good.
\end{enumerate} 
\pause
Correct answer: 1
}

\frame{
\frametitle{Question 19}
What is meant by the principle of ``subsidiarity''? [Please check the one correct answer.]
\begin{enumerate}
\item How many individuals and nations a public good affects.
\item That the political jurisdiction and the spillover range should match.
\item That people should contribute to a good based on their marginal willingness to pay, not their ability to pay.
\end{enumerate} 
\pause
Correct answer: 2
}

\frame{
\frametitle{Question 20}
How are transnational public goods financed? [Please check all the correct answers]
\begin{enumerate}
\item Contributions can be funnelled through multilateral or regional institutions.
\item A rich nation provides all the rest of the nations with the good.
\item Nations can combine efforts to provide good.
\item Taxation at a regional or global level.
\end{enumerate} 
\pause
Correct answers: 1 - 2 - 3
}

\frame{
\frametitle{Question 21}
What is political conditionality as it relates to foreign aid? [Please check the one correct answer.]
\begin{enumerate}
\item The requirement goods and services be purchased from firms in donor country,
or used for purposes that support groups in the donor countries.
\item The threat or action of reducing or terminating aid to mitigate human
rights abuse or to promote democracy.
\item Aid given on the condition that state-owned enterprises are privatized and that states impose fewer economic regulations.
\end{enumerate} 
\pause
Correct answer: 2
}

\frame{
\frametitle{Question 22}
What best describes the concept of ``herd immunity''? [Please check the one correct answer.]
\begin{enumerate}
\item One person's immunity has negative externalities on another person's health.
\item An immunized portion of the population provides protection to the
non-immunized.
\item When farmers are able to solve the tragedy of the commons for their grazing
cattle.
\item One persons disease has positive externalities on another person's health.
\end{enumerate} 
\pause
Correct answer: 2
}

\frame{
\frametitle{Question 23}
What best describes the concept of Transnational Public Good aid (TPG aid)? [Please check the one correct answer.]
\begin{enumerate}
\item Provision of TPGs for benefit of all (including donor).
\item A public good that rich countries can free-ride on.
\item Aid with no excludable benefits.
\item A rival public good.
\end{enumerate} 
\pause
Correct answer: 1
}

\frame{
\frametitle{Question 24}
What sort of market failure do patents for drugs represent? [Please check the one correct answer.]
\begin{enumerate}
\item Information asymmetry.
\item Externalities.
\item Failure of competition.
\item Pure public goods.
\end{enumerate} 
\pause
Correct answer: 3
}

\frame{
\frametitle{Question 25}
Which of the following are development traps, according to Paul Collier's book
``The Bottom Billion''? [Please check all the correct answers.]
\begin{enumerate}
\item Enduring conflict.
\item Not possessing sufficient natural resources.
\item Being landlocked with bad neighbours.
\item Experiencing bad governance in a large country.
\item Not enough investment.
\end{enumerate} 
\pause
Correct answers: 1 - 3
}

\end{document}